% Metódy inžinierskej práce

\documentclass[10pt,twoside,slovak,a4paper]{article}

\usepackage[slovak]{babel}
%\usepackage[T1]{fontenc}
\usepackage[IL2]{fontenc} % lepšia sadzba písmena Ľ než v T1
\usepackage[utf8]{inputenc}
\usepackage{graphicx}
\usepackage{url} % príkaz \url na formátovanie URL
\usepackage{hyperref} % odkazy v texte budú aktívne (pri niektorých triedach dokumentov spôsobuje posun textu)
\usepackage{cite}
%\usepackage{times}

\pagestyle{headings}


\title{Fake News Detection Using Machine Learning Approaches\thanks{Semestrálny projekt v predmete Metódy inžinierskej práce, ak. rok 2023/24, vedenie: Mohammad Yusuf Mamund}} % meno a priezvisko vyučujúceho na cvičeniach

\author{Šimon Podolan\\[2pt]
	{\small Slovenská technická univerzita v Bratislave}\\
	{\small Fakulta informatiky a informačných technológií}\\
	{\small \text{xpodolan@stuba.sk}}}
\date{\small 30. september 2023} % upravte
\begin{document}
    \maketitle
    
\begin{abstract}
The spread of fake news across social media and various media platforms has become a big issue, given its potential to inflict significant societal and national harm, often with far-reaching consequences. This paper starts by taking a close look at the research about finding fake news. After that, it explores the world of older machine learning methods.
The primary objective is to identify the most effective machine learning algorithm capable of classifying news articles as either authentic or fake. To accomplish this, the study uses machine learning techniques, employing tools such as Python's scikit-learn and Natural Language Processing. The goal of this research is to contribute to the ongoing efforts to combat the propagation of fake news by developing a robust and accurate machine learning model capable of distinguishing between authentic and fabricated information. This effort seeks to make us better at fighting the harmful effects of fake news on society and how we talk about important national matters.
\end{abstract}



\section{Intro}

In an era dominated by the rapid dissemination of information through social media and various media platforms, the proliferation of fake news has emerged as a formidable challenge. The implications of this alarming trend are vast, ranging from sowing discord within societies to potentially influencing critical national decisions. This paper embarks on a critical exploration of the ever-evolving battle against fake news.

Our journey commences with a deep dive into the extensive body of research dedicated to the detection of fake news. From there, we venture into the realm of traditional machine learning methods, seeking to unearth the most effective algorithm capable of distinguishing between authentic and counterfeit news articles. Our approach leverages the power of machine learning techniques, harnessing the formidable capabilities of tools such as Python's scikit-learn and Natural Language Processing.
At its core, this research endeavor aspires to make a profound contribution to the ongoing efforts to combat the propagation of fake news. The ultimate goal is to forge a robust and highly accurate machine learning model, one that can adeptly identify the deceptive narratives that threaten the integrity of information in our digital age. By doing so, we aim to bolster our collective ability to navigate the perilous landscape of fake news, safeguarding the way we discuss and deliberate upon vital national issues. In this pursuit, we strive to be better equipped to mitigate the detrimental impacts of misinformation on our society and the broader discourse of our most critical affairs.~\ref{zaver}.



\section{Nejaká časť} \label{nejaka}

Z obr.~\ref{f:rozhod} je všetko jasné. 

\begin{figure*}[tbh]
\centering
%\includegraphics[scale=1.0]{diagram.pdf}
Aj text môže byť prezentovaný ako obrázok. Stane sa z neho označný plávajúci objekt. Po vytvorení diagramu zrušte znak \texttt{\%} pred príkazom \verb|\includegraphics| označte tento riadok ako komentár (tiež pomocou znaku \texttt{\%}).
\caption{Rozhodujúci argument.}
\label{f:rozhod}
\end{figure*}



\section{Iná časť} \label{ina}

Základným problémom je teda\ldots{} Najprv sa pozrieme na nejaké vysvetlenie (časť~\ref{ina:nejake}), a potom na ešte nejaké (časť~\ref{ina:nejake}).\footnote{Niekedy môžete potrebovať aj poznámku pod čiarou.}

Môže sa zdať, že problém vlastne nejestvuje\cite{Coplien:MPD}, ale bolo dokázané, že to tak nie je~\cite{Czarnecki:Staged, Czarnecki:Progress}. Napriek tomu, aj dnes na webe narazíme na všelijaké pochybné názory\cite{PLP-Framework}. Dôležité veci možno \emph{zdôrazniť kurzívou}.


\subsection{Nejaké vysvetlenie} \label{ina:nejake}

Niekedy treba uviesť zoznam:

\begin{itemize}
\item jedna vec
\item druhá vec
	\begin{itemize}
	\item x
	\item y
	\end{itemize}
\end{itemize}

Ten istý zoznam, len číslovaný:

\begin{enumerate}
\item jedna vec
\item druhá vec
	\begin{enumerate}
	\item x
	\item y
	\end{enumerate}
\end{enumerate}


\subsection{Ešte nejaké vysvetlenie} \label{ina:este}

\paragraph{Veľmi dôležitá poznámka.}
Niekedy je potrebné nadpisom označiť odsek. Text pokračuje hneď za nadpisom.



\section{Dôležitá časť} \label{dolezita}




\section{Ešte dôležitejšia časť} \label{dolezitejsia}




\section{Záver} \label{zaver} % prípadne iný variant názvu



%\acknowledgement{Ak niekomu chcete poďakovať\ldots}


% týmto sa generuje zoznam literatúry z obsahu súboru literatura.bib podľa toho, na čo sa v článku odkazujete
\bibliography{literatura}
\bibliographystyle{plain} % prípadne alpha, abbrv alebo hociktorý iný
\end{document}
